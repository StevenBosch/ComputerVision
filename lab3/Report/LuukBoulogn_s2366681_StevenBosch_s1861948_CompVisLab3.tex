\documentclass{article}
\usepackage[
        a4paper,% other options: a3paper, a5paper, etc
        left=3cm,
        right=3cm,
        top=3cm,
        bottom=4cm,
        % use vmargin=2cm to make vertical margins equal to 2cm.
        % us  hmargin=3cm to make horizontal margins equal to 3cm.
        % use margin=3cm to make all margins  equal to 3cm.
]{geometry}
%\usepackage[utf8x]{inputenc}
\usepackage{graphicx}
\usepackage{caption}
\usepackage{enumerate}
\usepackage{subcaption}
\usepackage[procnames]{listings}
\usepackage{color}
\usepackage{amssymb}
\usepackage{amsmath}
\usepackage{comment}
\usepackage{hyperref}
\usepackage{blindtext}
\usepackage[titletoc,title]{appendix}
\usepackage{float}
\usepackage{fullpage}
\definecolor{codegreen}{rgb}{0,0.6,0}
\definecolor{codegray}{rgb}{0.5,0.5,0.5}
\definecolor{codepurple}{rgb}{0.58,0,0.82}
\definecolor{backcolour}{rgb}{0.95,0.95,0.92}

\lstdefinestyle{mystyle}{
    backgroundcolor=\color{backcolour},
    commentstyle=\color{codegreen},
    keywordstyle=\color{magenta},
    numberstyle=\tiny\color{codegray},
    stringstyle=\color{codepurple},
    basicstyle=\ttfamily,
    breakatwhitespace=false,
    breaklines=true,
    captionpos=t,
    keepspaces=true,
    numbers=left,
    numbersep=5pt,
    showspaces=false,
    showstringspaces=false,
    showtabs=false,
    tabsize=2
}

\lstset{style=mystyle, language=Matlab}

\renewcommand{\thesection}{\large{Exercise \arabic{section}}}

\title{Computer Vision - Lab 3}
\author{Luuk Boulogne (s2366681) \and Steven Bosch (s1861948)}
\date{\today}

\begin{document}
\maketitle

\section{}
The exercise defines that the figure represents rectangles in 3D.
The perspective projections of the edges of these rectangles can be represented as linear functions ($ax+b$), which can be acquired by sampling two points on the lines.

Since we have two sets of parallel lines in 3D, we can find two vanishing points on the image plane, $(u_{\infty 1},v_{\infty 1})$ and $(u_{\infty 2}, v_{\infty 2})$, by finding the intersection of their respective linear functions.

Using these vanishing points, the two direction vectors $\vec{l}$ and $\vec{k}$ are obtained according to the equation described in section 4.2.1.:

\begin{equation}
\left(
\begin{array}{c}
l_1 \\ l_2 \\ l_3
\end{array}
\right)
= \frac{1}{\sqrt{u_{\infty 1}^2+v_{\infty 1}^2+f^2}}
\left(
\begin{array}{c}
u_{\infty 1} \\ v_{\infty 1} \\ f
\end{array}
\right)
\end{equation}
and 

\begin{equation}
\left(
\begin{array}{c}
k_1 \\ k_2 \\ k_3
\end{array}
\right)
= \frac{1}{\sqrt{u_{\infty 2}^2+v_{\infty 2}^2+f^2}}
\left(
\begin{array}{c}
u_{\infty 2} \\ v_{\infty 2} \\ f
\end{array}
\right)
\end{equation}

Since we know that both of the direction vectors of the two sets of parallel lines are perpendicular to one another, we know that their dot-product has to be zero:

\begin{align}
\vec{l} \cdot \vec{k}  = 0 \\
\frac{1}{\sqrt{u_{\infty 1}^2+v_{\infty 1}^2+f^2}}
\left(
\begin{array}{c}
u_{\infty 1} \\ v_{\infty 1} \\ f
\end{array}
\right) \cdot \frac{1}{\sqrt{u_{\infty 2}^2+v_{\infty 2}^2+f^2}}
\left(
\begin{array}{c}
u_{\infty 2} \\ v_{\infty 2} \\ f
\end{array}
\right)  = 0 \\
\frac{1}{\sqrt{u_{\infty 1}^2+v_{\infty 1}^2+f^2}} * \frac{1}{\sqrt{u_{\infty 2}^2+v_{\infty 2}^2+f^2}} (u_{\infty 1}u_{\infty 2}+v_{\infty 1}v_{\infty 2} + f^2) = 0 \\
u_{\infty 1}u_{\infty 2}+v_{\infty 1}v_{\infty 2} + f^2 = 0 \\
f = \sqrt{-(u_{\infty 1}u_{\infty 2}+v_{\infty 1}v_{\infty 2})}
\end{align}
Now that we have determined $f$, we can calculate the direction vectors $\vec{l}$ and $\vec{k}$ as described in equations 1 and 2 above. Finally we can determine the normal of the planar patch containing the rectangles as the cross product of the direction vectors: $\vec{l} \times \vec{k}$. 

\section{}
We have created the files containing the scanpoint of the parallel lines.

\section{}
The function \texttt{par\_line.m} calculates the direction vector of a given set of points that represent parallel lines according to the method given in section 4.2.2.

\section{}
We wrote three distinct functions for this exercise. The function \texttt{myRectangle} reads in the sets of points using the provided function \texttt{readfile}. It then calls upon the function \texttt{vanishPoint} to calculate the vanishing points of the sets of points. \texttt{vanishPoint} receives one set of lines that are parallel in 3D and calculates its vanishing point in the following manner:

\begin{enumerate}
 \item For every set of two points describing a line, its linear function is calculated as $y = ax + b$, where $a$ is calculated as $a = \frac{y_1-y_2}{x_1-x_2}$ and b as $b = y_1 - a*x_1$.
 \item The intersection of all combinations of two linear functions are calculated as follows: $x = \frac{b_1 - b_2}{a_2 - a_1}$ and $y = a_1*x + b_1$.
 \item Finally the median of all these points is returned as vanishing point. We use the median as it is robust to outliers (inaccurate clicks) and therefore gives the best representation of the actual vanishing points.
\end{enumerate}

With the acquired vanishing points the camera constant $f$ is determined as described in exercise 1, equation 7. Subsequently the function \texttt{dirVector} is called on with a data set of points and the calculated camera constant. In \texttt{dirVector} the direction vector of the given set of points is calculated in the same way as in \texttt{par\_line.m} and returned.

Finally the normal of the planar patch containing the rectangles is calculated as the cross product of the two direction vectors.

\subsection{Output}
We created two files containg points and call upon the function using \texttt{myRectangle('par\_lines.dat', 'par\_lines2.dat')}. It yields the following output:

\begin{lstlisting}
The camera constant f = 1201.211921
The first direction vector: (-0.626905 0.578391 0.521971)
The second direction vector: (0.807287 0.386091 0.446341)
The normal of the planar patch is: (0.056632 0.701194 -0.708971)
\end{lstlisting}

\end{document}