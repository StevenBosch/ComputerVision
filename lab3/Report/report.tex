\documentclass{article}
\usepackage[
        a4paper,% other options: a3paper, a5paper, etc
        left=3cm,
        right=3cm,
        top=3cm,
        bottom=4cm,
        % use vmargin=2cm to make vertical margins equal to 2cm.
        % us  hmargin=3cm to make horizontal margins equal to 3cm.
        % use margin=3cm to make all margins  equal to 3cm.
]{geometry}
%\usepackage[utf8x]{inputenc}
\usepackage{graphicx}
\usepackage{caption}
\usepackage{enumerate}
\usepackage{subcaption}
\usepackage[procnames]{listings}
\usepackage{color}
\usepackage{amssymb}
\usepackage{amsmath}
\usepackage{comment}
\usepackage{hyperref}
\usepackage{blindtext}
\usepackage[titletoc,title]{appendix}
\usepackage{float}
\usepackage{fullpage}
\definecolor{codegreen}{rgb}{0,0.6,0}
\definecolor{codegray}{rgb}{0.5,0.5,0.5}
\definecolor{codepurple}{rgb}{0.58,0,0.82}
\definecolor{backcolour}{rgb}{0.95,0.95,0.92}

\lstdefinestyle{mystyle}{
    backgroundcolor=\color{backcolour},
    commentstyle=\color{codegreen},
    keywordstyle=\color{magenta},
    numberstyle=\tiny\color{codegray},
    stringstyle=\color{codepurple},
    basicstyle=\ttfamily,
    breakatwhitespace=false,
    breaklines=true,
    captionpos=t,
    keepspaces=true,
    numbers=left,
    numbersep=5pt,
    showspaces=false,
    showstringspaces=false,
    showtabs=false,
    tabsize=2
}

\lstset{style=mystyle, language=Matlab}

\renewcommand{\thesection}{\large{Exercise \arabic{section}}}

\title{Computer Vision - Lab 3}
\author{Luuk Boulogne (s2366681) \and Steven Bosch (s1861948)}
\date{\today}

\begin{document}
\maketitle

\section{}
The exercise defines that the figure represents rectangles in 3D.
The perspective projections of the edges of these rectangles can be represented as linear functions ($ax+b$), which can be acquired by sampling two points on the lines.

Since we have two sets of parallel lines in 3D, we can find two vanishing points on the image plane, $(u_{\infty 1},v_{\infty 1})$ and $(u_{\infty 2}, v_{\infty 2})$, by finding the intersection of their respective linear functions.

Using these vanishing points, the two direction vectors $\vec{l}$ and $\vec{k}$ are obtained according to the equation described in section 4.2.1.:

\begin{equation}
\left(
\begin{array}{c}
l_1 \\ l_2 \\ l_3
\end{array}
\right)
= \frac{1}{\sqrt{u_{\infty 1}^2+v_{\infty 1}^2+f^2}}
\left(
\begin{array}{c}
u_{\infty 1} \\ v_{\infty 1} \\ f
\end{array}
\right)
\end{equation}
and 

\begin{equation}
\left(
\begin{array}{c}
k_1 \\ k_2 \\ k_3
\end{array}
\right)
= \frac{1}{\sqrt{u_{\infty 2}^2+v_{\infty 2}^2+f^2}}
\left(
\begin{array}{c}
u_{\infty 2} \\ v_{\infty 2} \\ f
\end{array}
\right)
\end{equation}

Since we know that both of the direction vectors of the two sets of parallel lines are perpendicular to one another, we know that their dot-product has to be zero:

\begin{align}
\vec{l} \cdot \vec{k}  = 0 \\
\frac{1}{\sqrt{u_{\infty 1}^2+v_{\infty 1}^2+f^2}}
\left(
\begin{array}{c}
u_{\infty 1} \\ v_{\infty 1} \\ f
\end{array}
\right) \cdot \frac{1}{\sqrt{u_{\infty 2}^2+v_{\infty 2}^2+f^2}}
\left(
\begin{array}{c}
u_{\infty 2} \\ v_{\infty 2} \\ f
\end{array}
\right)  = 0 \\
\frac{1}{\sqrt{u_{\infty 1}^2+v_{\infty 1}^2+f^2}} * \frac{1}{\sqrt{u_{\infty 2}^2+v_{\infty 2}^2+f^2}} (u_{\infty 1}u_{\infty 2}+v_{\infty 1}v_{\infty 2} + f^2) = 0 \\
u_{\infty 1}u_{\infty 2}+v_{\infty 1}v_{\infty 2} + f^2 = 0 \\
f = \sqrt{-(u_{\infty 1}u_{\infty 2}+v_{\infty 1}v_{\infty 2})}
\end{align}
Now that we have determined $f$, we can calculate the direction vectors $\vec{l}$ and $\vec{k}$ as described in equations 1 and 2 above. Finally we can determine the normal of the planar patch containing the rectangles as the cross product of the direction vectors: $\vec{l} \times \vec{k}$.

\section{}
We have created the files containing the scanpoint of the parallel lines.

\stepcounter{section}

\section{}


\end{document}