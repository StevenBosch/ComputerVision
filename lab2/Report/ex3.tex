ex3
When using the classical snake, the shape of the lungs is not approximated well for any parameter setting, without smoothing the image first. When the image is smoothed with $\sigma$ = 3, the left edge of the left lung and almost the entire right lung are approximated well. This is because the smoothing causes the radius in which an edge causes force vectors to point towards it to become larger. Even though the entire snake is initialized within the lung and therefore the internal forces do not automatically cause the snake to converge towards the shape, when the image is smoothed external forces are enough in this case to propegate the nodes of snake towards the edges.

With GVF, the shape of both lungs can be approximated quite well, even without smoothing. This is because now, external forces cause the snake to converge to the shape of the lungs, because of the first term of equation 12 in the paper. There are some edges within the left lung that some nodes of the snake stick to, especially when no smoothing is applied. In the right lung, the bottom edge of the lung is ignored. This is because of the large force caused by the sharper edge below this bottom edge.


ex4
Again, the classical snake did not find a good solution without smoothing. With smoothing, for the right lung, the right side of the shape is approximated better in the high contrast image, with respect to the original image. This is probably due to the sharper edge. On the left side of the right lung, the shape of the lung is approximated more poorly in the high contrast image. This is due to the detail edges inside the lung that the nodes in the snake stick to.

With GVF, also without smoothing solutions were found for both lungs. For the left lung, the performance was similar. For the right lung, the performance was better in the bottom part, since for the most part, the actual shape of the lung was found instead of the edge below it. However, there are also some new locations on the shape of the lung in which now an edge outside of the lung is found instead of the edge of the lung itself, where this was not the case with the original image.

When in the images of exercise 3 and 4, beta can be set to a non zero value (e.g. 0.2) to deal with the lines with a dead end that go inside the lung. In this case the overall shape detection does not alter, only the lines going inward with a dead end are deleted.

ex6
A positive side of this method is that it delivers a digital contour that can be used as input for other software. Also, with human assistence the method is able to approximate contours neatly with only little effort of the human. On the other hand, this is also the method's main pitfall: it is unable to do anything properly without human intervention.

We therefore think that the key problem of snake-based segmentation is that it cannot work autonomously. The reasons for this are the following.
First of all, snakes needs to be initialized in a way so that it can find the shape. Therefore snake-based segmentation cannot work autonomously for new images. 
Also, there are many parameters that should be tweaked to find the best solution that the method can produce. Optimal parameter settings are not known for new images and thus the optimal solution cannot be found without human intervention.
Finally, there is no given performance measure for the optimality of the solution. Again, a human has to evaluate the reached solution.

The two methods can be combined in a way to get better results. For example with 'room.pgm', the classical snake only finds a good solution if the entire room is inside the snake at the initialisation. The GVF approach does always find a good solution, but it doesn't represent the corners very well. By first letting the GVF snake find a solution and afterwards optimizing this solution with the classical snake, the better solution can be found even if not the entire room is inside the snake at the initialisation.